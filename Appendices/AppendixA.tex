% Appendix A

\begin{appendices}
\chapter{{\sc collr-v1.0}} % Main appendix title

\label{app:radt} % For referencing this appendix elsewhere, use \ref{AppendixA}

\lhead{\emph{Appendix A} \emph{{\sc collr-v1.0}}} % This is for the header on each page - perhaps a shortened title

We provide a working code for the computer package {\sc collr-v1.0}, and this can be implemented alongside the test case and \textit{adf04} file as described in Section \ref{sec:spe_imp}. 

\section*{Namelist variables}
We have provided a sample \textit{rad.inp} test case already in Section \ref{sec:spe_imp}, and now for completeness, we also provide a description of the variables mentioned in the namelists \texttt{COLLINP} and \texttt{LIRA}.

\begin{itemize}
\item{\texttt{corrap = 1} computes the coronal approximation in equation (\ref{eq:spe_coronal}) for the temperatures considered in the \textit{adf04} file, and results are printed at the end of \textit{coll.out}. The values are not computed and can be switched off by setting \texttt{= 0}.}

\item{\texttt{lte} is similar to \texttt{corrap} in terms of input and output, but now by using equation (\ref{eq:spe_lte}).}

\item{\texttt{notemp} is equal to the number of header temperatures appearing in the \textit{adf04} input file, and a subset of these can be considered by setting \texttt{mxtemp} $\geq$ \texttt{mntemp} $\geq$ 1}

\item{\texttt{Nodens > 2} controls the number of densities to be included, and is bounded by 10$^{\texttt{sDens}} \leq N_e \leq 10^{\texttt{fDens}}$.}

\item{\texttt{grad = 1} reads the $A$-values from a separate \textit{adf04rad} input file as `upper lvl  \dots lower lvl \dots $A$-values', and replaces those in the \textit{adf04} file. The values directly from the \textit{adf04} file are implemented if \texttt{= 0}.}

\item{\texttt{IC = 1} defines the coupling scheme, and the statistical weight is $g_i = 2J_i+1$, and similarly if \texttt{= 0}, then $LS\pi$ coupling is assumed.}

\item{\texttt{pops = 1} prints the populations as a function of the electron density for each temperature (each column) and the resulting output is printed to \textit{population\_i} for each level $i$.}

\item{\texttt{petol = -2} controls the printout of the photon emissivity coefficients, and only those that are stronger than $\geq 10^{\texttt{petol}}$ are printed to \textit{PEC.out}.}

\item{\texttt{wrngdn / wrngup} defines the wavelength region in {\rm \AA} that is allowed to print to \textit{PEC.out} and is usually set to unity for \texttt{wrngdn} and a large value for \texttt{wrngup} (Only used if \texttt{ppec = 0}).}

\item{\texttt{uplev / uplevl / lolev / lolevl} are the indicies required to compute a particular line ratio given by $\mathcal{R}$ = $I_{\texttt{uplev}\rightarrow\texttt{uplevl}}$/$I_{\texttt{lolev}\rightarrow\texttt{lolevl}}$ }

\item{\texttt{ppec = 1} is for a particular photon emissivity coefficient defined by \texttt{uppec} $\rightarrow$ \texttt{lopec}, and it gets written into the output file \textit{PEC.out}. It is recorded for every temperature and density.}

\end{itemize}



\section*{{\sc fortran} code}
All the results from Figures \ref{fig:spe_sxb1}, \ref{fig:spe_sxb2}, \ref{fig:co_populations}, \ref{fig:co_pop2}, \ref{fig:co_consttemp}, and \ref{fig:co_constdens}, have been computed with the code below.

\begin{verbatim}
                            program CollR
!
  implicit real*8(A-H,O-Z)
!
  NAMELIST/COLLINP/Nodens,notemp,sDens,fDens,IC,popt,grad, &
          & worder,petol,mntmp,mxtmp,wrngup,wrngdn,pops,corapp,lte
  NAMELIST/LIRA/uplev,uplevl,lolev,lolevl,uppec,lopec,ppec
!
  integer :: i,ii,j,jj,jjj,k,kk,kkk,order,Ne,prnt,pops, &
          &  corapp,popt,IC,grad,icount,worder,dum,dumm, &
          &  einf,jcount,levels,notemp,Z,tripi,tripj,tk,Nee, &
          &  Nodens,sDens,fDens,uplev,lolev,kcount,petol, &
          &  lopec,uppec,dumrr,ppec,mntmp,mxtmp,cup, &
          &  clo,wrngup,wrngdn,uplevl,lolevl,lte,collprint

  real*8 :: hold,IH,dumr,start,finish,start1,finish1, &
          &  start2,finish2,start3,finish3,petolr, &
          &  const,boltz,factor,plank,sol,Me,zero, &
          &  newDen,corappv,ltev,DE1,DE2,totAup,totAlo

  real*8, allocatable :: aij(:,:),rate(:,:,:),temp(:), &
          &  holde(:),ED(:),XJ(:),weight(:),en(:),err(:), &
          &  lratio(:,:),COLL(:,:,:,:),LHS(:,:,:), &
          &  pop(:,:,:),wlength(:,:), &
          &  pec(:,:,:,:),wout(:)

  integer, allocatable :: ind(:),mult(:),bEll(:), &
          & ipoint(:), jpoint(:)

  character*13, allocatable :: config(:)

  character*3 :: charin,chr
!
  parameter(MXEN=999)         ! NUMBER OF TARGET STATES
!
  open(15, file=`rad.inp',status=`unknown', form=`formatted')
  open(16, file=`rad.out',status=`unknown', form=`formatted')
  open(404, file=`const_dens',status=`unknown', form=`formatted')
  open(405, file=`const_temp',status=`unknown', form=`formatted')
!
  call cpu_time(start)
  call cpu_time(start1)
!
  worder = 0
  corapp = 0
    pops = 0
   petol = 0
   lopec = 1
   uppec = 2
    ppec = 0
   mntmp = 1
   mxtmp = -1
  wrngup = 0
  wrngdn = 0
    line = 0
    grad = 0
    popt = 0
     clo = 0
     cup = 0
    einf = 0
!
  read(15,collinp)
  read(15,lira)
  if (mxtmp.eq.-1) then
    mxtmp = notemp
  endif
!
  if ((Nodens.lt.2).or.(Nodens.gt.999)) then
      print *, `Warning, Nodens in radinp must be 2 &
               & < Nodens < 999'
  endif
!
! Electron density mesh
!
  allocate(ED(Nodens))
!
  newDen = real(sDens)
!
  if (Nodens.eq.2) then
     ED(1) = 10.0d0**(real(sDens))
     ED(2) = 10.0d0**(real(fDens))
  else
        if ((real(fDens) - real(sDens).lt.1.1)) then
           Me = ((real(fDens) - real(sDens)))/ &
                & (real(Nodens) - 1.0)
        else
           Me = ((real(fDens) - real(sDens)))/ &
                & (real(Nodens) - 1.0)
        endif
     ED(1) = 10.0d0**(real(sDens))
        do i=2,Nodens
           newDen = newDen + Me
           ED(i) = 10.0d0**(real(newDen))
           print *, `Density: ', i, ED(i)
        enddo
  endif
!
  close(15)
!
! Input constants
!
  const = 2.1716d-8          ! 2pia_0 const. for rates (q_ij)
     IH = 13.605698          ! Ionization of Hydrogen
  boltz = 8.61733247d-5      ! Boltzmann constant in eV
  plank = 4.13566766225d-15  ! Plancks constant to convert to Ang
    sol = 2.99792458d8       ! Speed of light
!
  open(10, file =`adf04', status = `old')
  read(10,1000) Z
!
  levels = 0
!
! Calculate the number of levels in adf04 file
!
  do i=1,MXEN
     read(10,*) dum
        if(dum.eq.(-1))then
           go to 200
        else
           continue
        endif
     levels = levels + 1
  enddo
!
! Reset reading of input back to the first level
!
  200 continue
     do i=1,(levels+1)
        backspace(10)
     enddo
!
  allocate(ind(levels))
    allocate(en(levels))
      allocate(mult(levels))
        allocate(bEll(levels))
          allocate(XJ(levels))
            allocate(weight(levels))
            allocate(config(levels))
          allocate(aij(levels,levels))
        allocate(rate(levels,levels,notemp))
      allocate(temp(notemp))
    allocate(holde(notemp+einf))
  allocate(err(levels))
!
  write(16,7000)
  write(16,7001) Z, notemp, Nodens, sDens, fDens, Me
  write(16,7002)
  write(16,7003) levels
!
! Calculate rates from Upsilons in ADF04
!
  do i=1,levels
  read(10,2001) ind(i), config(i), mult(i), bEll(i), XJ(i), &
             &  en(i)
!
        if (i.eq.levels) then
           if (ind(i).eq.levels) then
             print *, `Read of levels correct, total in &
                    &  ADF04: ', levels
           else
             print *, `Read of crucial input data INCORRECT.'
           endif
        endif
!
! cm-1 -> eV conversion
     en(i) = en(i)*0.000123984
     write(16,7004), ind(i), config(i), en(i)
! Statistical weightings..
! LSpi / Jpi :
        if (IC.eq.0) then
           weight(i) = (mult(i))*(2*bEll(i)+1)
        else
           weight(i) = (2*XJ(i)+1)
        endif
  enddo
!
  read(10,*) dumm
  read(10,*) dumr, dumrr, (temp(i), i=1,(notemp))
  print *, `Check temperatures are correct:', &
        &  (temp(i), i=1,(notemp))
!
! tripi/j, hold and holde used for holding variables
!
  if (grad.eq.1) then
  open(5,file=`adf04rad',status=`unknown', form=`formatted')
  endif
!
   aij = 0.0
  rate = 0.0
!
  order = ((levels**2-levels)/2)
!
  do i=1,order
     read(10,2003) tripj, tripi, hold, (holde(j), &
                &  j=1,notemp+einf)
!
        if (grad.eq.1) then
           read(5,*) dumrr, dumm, hold
        endif
        if (hold.lt.1e-20) then
           hold = 0.0
        endif
!
    aij(tripj,tripi) = hold
    aij(tripi,tripj) = hold
!
! Excitation rate depends on T only
!
     do k=1,notemp
      rate(tripi,tripj,k) = (const/weight(tripi))* &
           & sqrt((IH)/(boltz*temp(k)))* &
           & exp((en(tripi)-en(tripj))/ &
           & (boltz*temp(k)))*holde(k)
!
! De-excitation rate.
!
      rate(tripj,tripi,k) = (weight(tripi)/weight(tripj))* &
           & exp(-(en(tripi)-en(tripj))/(boltz*temp(k)))* &
           & rate(tripi,tripj,k)
     enddo
!
  enddo

  DE1 = en(uplev)-en(uplevl)
  DE2 = en(lolev)-en(lolevl)
!
! Important formats for ADF04 file
! These may need changed depending
! on the format of the adf04 file.
!
  1000 format(13X,I2)
  1001 format(2X,I3,3X,A13,4X,I1,1X,I1,1X,F4.1,3X,F9.1)
  1003 format(2X,I2,2X,I2,25F8.5)
!
  2001 format(2X,I3,3X,A13,4X,I1,1X,I1,1X,F4.1,4X,F8.1)
  2003 format(1X,I3,1X,I3,25F8.5)
  2010 format(1X,I3,1X,I3,1X,16F8.6)
!
     if (allocated(mult)) deallocate(mult)
     if (allocated(bEll)) deallocate(bEll)
     if (allocated(XJ)) deallocate(XJ)
!
! Read complete.
! A-values stored in aij(i,j)
! and rates in rate(i,j,k)
!
!        ####################################
!        # Fill the Collision matrix (COLL) #
!        ####################################
!
  allocate(COLL(levels,levels,notemp,Nodens))
  COLL = 0.0d0
!
print *, `Begin filling collision matrix..'
  do Ne=1,Nodens
     edens = edens + Me
     print *, Ne, edens
        do k=mntmp,mxtmp
           do i=1,levels
              do j=1,levels
                 if (i.eq.j) then
                    do kk=1,j
                       COLL(i,j,k,Ne) = COLL(i,j,k,Ne) - &
                                      & aij(j,kk)
                    enddo
                    do kkk=1,levels
                       if (kkk.eq.j) then
                          go to 99
                       else
                          COLL(i,j,k,Ne) = COLL(i,j,k,Ne) - &
                          & ((ED(Ne))*(rate(j,kkk,k)))
                       endif
                          99 continue
                    enddo
                 else if (i.gt.j) then
                    COLL(i,j,k,Ne) = COLL(i,j,k,Ne) + &
                    & ((ED(Ne))*(rate(j,i,k)))
                 else if (i.lt.j) then
                    COLL(i,j,k,Ne) = COLL(i,j,k,Ne) + &
                    &  aij(i,j) + ((ED(Ne))*(rate(j,i,k)))
                 endif
              enddo
           enddo
        enddo
  enddo
!
  call cpu_time(finish1)
  print `("Basic Read = ",f6.3," seconds.")',finish1-start1
  call cpu_time(start2)
!
!        ########################
!        # Gaussian Elimination #
!        ########################
!
  allocate(LHS((levels),notemp,Nodens))
!
  LHS = 0.0d0
!
  do k=mntmp,mxtmp
     do kk=1,Nodens
        do kkk=1,(levels)
           LHS(kkk,k,kk) = -COLL((kkk),1,k,kk)
        enddo
     enddo
  enddo
!
  print *, `Begin Guassian elimination..'
  print *, `Density      Temperature'
  print *, `________________________'
!
  do Ne=1,Nodens
     print *,`  ', Ne
        do kk=mntmp,mxtmp
           print *, `           ', kk
              do i=2,(levels-1)
                 do j=i+1,levels
                    factor=COLL(j,i,kk,Ne)/COLL(i,i,kk,Ne)
                    COLL(j,i,kk,Ne)=0.0d0
                    LHS(j,kk,Ne) = LHS(j,kk,Ne) - &
                                 &  factor*LHS(i,kk,Ne)
                       do k=i+1,levels
                          COLL(j,k,kk,Ne) = COLL(j,k,kk,Ne) &
                                &  - factor*COLL(i,k,kk,Ne)
                       enddo
                 enddo
              enddo
        enddo
  enddo
!
  call cpu_time(finish2)
  print `("Gaussian elimination = ",f6.3," seconds.")', &
        &  finish2-start2
!
!
  write(16,7005) levels
  jcount = 0
!
  do k=2,levels
     if (mod(k,8).eq.0) then
     write(16,7006) (LHS(kk+(jcount*8),mntmp,1), kk=1,8)
        jcount = jcount + 1
     else if (k.eq.levels) then
     write(16,7006) (LHS(levels-mod(levels,8)+kk,mntmp,1), &
        & kk=1,(mod(levels,8)))
     else
        continue
     endif
  enddo
!
!        #############################
!        # Populations & Line ratios #
!        #############################
!
  icount = 0
  allocate(pop(levels,notemp,Nodens))
  pop = 0.0
!
  do Ne=1,Nodens
     do kk=mntmp,mxtmp
        do i=levels,2,-1
           pop(i,kk,Ne) = (LHS(i,kk,Ne))/(COLL(i,i,kk,Ne))
              if (i.ne.levels) then
                 do j=levels,(i+1),-1
                    pop(i,kk,Ne) = pop(i,kk,Ne) - &
                    &  ((COLL(i,j,kk,Ne)*pop(j,kk,Ne))/ &
                    &  (COLL(i,i,kk,Ne)))
                 enddo
              endif
        enddo
     enddo
  enddo
!
! Plot the populations if pops = 1
!
  if (pops.eq.1) then
     do i=1,levels
        write(charin,`(I3)') i
        charin = adjustl(charin)
        open (410 + i, file=`population_'//trim(charin), &
        & status=`unknown', form=`formatted')
            write(410 + i,7011) (temp(j), j=mntmp,notemp)
           do Ne=1,Nodens
            write(410 + i,7012) ED(Ne), (pop(i,k,Ne), k=mntmp,mxtmp)
           enddo
            close(410+i)
     enddo
  endif
!
   allocate(lratio(Nodens,notemp))
!
! Keeping Density constant...
!
  write(16,7007) uplev, uplevl, lolev, lolevl
  prnt = 0
!
  do Ne=1,Nodens
     if (prnt.eq.1) then
        write(16,7008) Ne, ED(Ne)
           do kk=mntmp,mxtmp
              lratio(Ne,kk) = (pop(uplev,kk,Ne)* &
              &  aij(uplev,uplevl))/ &
              &  (pop(lolev,kk,Ne)*aij(lolev,lolevl))
           write(16,7015) temp(kk), lratio(Ne,kk)*(DE1/DE2)
           enddo
     elseif (prnt.eq.0) then
        if (Ne.eq.1) then
           write(16,*) `-> Increasing density'
           write(16,7013) (ED(jj), jj=1,Nodens)
        else
           continue
        endif
!
           do kk=mntmp,mxtmp
              lratio(Ne,kk) = (pop(uplev,kk,Ne)* &
              & aij(uplev,uplevl))/ &
              & (pop(lolev,kk,Ne)*aij(lolev,lolevl))
           enddo
!
        if (Ne.eq.Nodens) then
           do kk=mntmp,mxtmp
              write(16,7012) temp(kk), &
              & (lratio(ii,kk)*(DE1/DE2), ii=1,Nodens)
              write(404,*) temp(kk), &
              & (lratio(ii,kk)*(DE1/DE2), ii=1,Nodens)
           enddo
        endif
     endif
  enddo
!
! Keeping Temperature constant:
!
  do i=1,2
     write(16,*)
  enddo
!
  do kk=mntmp,mxtmp
     if (prnt.eq.1) then
        write(16,7010) kk, temp(kk)
           do Ne=1,Nodens
              lratio(Ne,kk) = (pop(uplev,kk,Ne)* &
              &  aij(uplev,uplevl))/ &
              &  (pop(lolev,kk,Ne)*aij(lolev,lolevl))
              write(16,7015) ED(Ne), lratio(Ne,kk)*(DE1/DE2)
           enddo
           if (kk.eq.mxtmp) then
              write(16,*)
              write(16,*) `----END LINE RATIOS----'
              write(16,*)
           endif
     elseif (prnt.eq.0) then
        do Ne=1,Nodens
           lratio(Ne,kk) = (pop(uplev,kk,Ne)*  &
           &  aij(uplev,uplevl))/ &
           &  (pop(lolev,kk,Ne)*aij(lolev,lolevl))
        enddo
           if (kk.eq.mntmp) then
              write(16,*) `-> Increasing temperature'
              write(16,7011) (temp(jj), jj=mntmp,mxtmp)
           else
              continue
           endif
           !
           if (kk.eq.notemp) then
              do Ne=1,Nodens
           write(16,7012) ED(Ne), (lratio(Ne,ii)*(DE1/DE2), &
           &  ii=mntmp,mxtmp)
          write(405,*) ED(Ne), (lratio(Ne,ii)*(DE1/DE2), &
           &  ii=mntmp,mxtmp)
              enddo
                 write(16,*)
                 write(16,*) `----END LINE RATIOS----'
                 write(16,*)
           endif
          !
     endif
  enddo
!
!        ########################
!        #    Coronal + LTE     #
!        ########################
!
  if (lte.eq.1) then
     write(16,*) `LTE values (High Ne): '
        do k=mntmp,mxtmp
           ltev = (weight(uplev)/weight(lolev))* &
           & exp(-(en(uplev)-en(lolev))/(boltz*temp(k)))
         write(16,7014) k, temp(k), ltev*((aij(uplev,uplevl))/ &
                        &  (aij(lolev,lolevl)))*(DE1/DE2)
        enddo
  endif

  if (corapp.eq.1) then
!
  totAup = 0.0
  totAlo = 0.0
!
    write(16,*) `Coronal values (Low Ne):'

    do i=1,(uplev-1)
      totAup = totAup + aij(uplev,i)
    enddo
!
    do i=1,(lolev-1)
      totAlo = totAlo +  aij(lolev,i)
    enddo
!
     do k=mntmp,mxtmp
        corappv = (rate(1,uplev,k)/rate(1,lolev,k))* &
         &  (totAlo/totAup)* &
         &  (aij(uplev,uplevl)/aij(lolev,lolevl))
         write(16,7014) k, temp(k),(corappv*(DE1/DE2))
     enddo
  endif
!
  call cpu_time(start3)
!
!        ########################
!        # Line Identifications #
!        ########################
!
  petolr = real(10.0**(real(petol)))
!
    allocate(wlength(levels,levels-1))
    allocate(pec(levels,levels-1,notemp,Nodens))
!
  open(18,file=`lineID',status=`unknown',form=`formatted')
  write(18,*) ` U(j) L(i)   Up-Config &
  &       Lo-Config   Up (eV)  Lo (eV)  wlength (Ang)'
!
  open(19,file=`PEC_plot.out',status=`unknown',form=`formatted')
!
  do kk=mntmp,mxtmp
     if ((ppec.eq.0).and.(worder.eq.0)) then
        write(19,*) `Temperature: ', kk
        write(19,*) `--------------------'
        write(19,*)  ` Wavelength     Ind      Ne           PEC'
     endif
        do Ne=1,Nodens
           do j=2,levels
              do i=1,j-1
                 if ((kk.eq.mntmp).and.(Ne.eq.1)) then
            wlength(j,i) = (sol*plank)/(abs(en(j)-en(i)))
            wlength(j,i) = wlength(j,i)*1d10
            write(18,7016) j, i, config(j), config(i), &
                      & en(j), en(i), wlength(j,i)
                 endif
                    pec(j,i,kk,Ne) = pop(j,kk,Ne)*aij(j,i)
!
! Check the following:
!               1) wavelength ordering is off
!               2) ppec is for all PEC's.
!               3) Only PEC's above a certain tolerance
!               4) Only wavelengths in a certain range
        if (worder.eq.0) then
          if (ppec.eq.0) then
            if ((pec(j,i,kk,Ne)).gt.petolr) then
              if ((wrngup.gt.0).or.(wrngdn.gt.0)) then
                if ((wlength(j,i).gt.wrngdn).and. &
                   & (wlength(j,i).lt.wrngup)) then
                      write(19,7017) wlength(j,i), Ne, ED(Ne), &
                                   & pec(j,i,kk,Ne)
                endif
              endif
            endif
          endif
        endif
                 !
              enddo
           enddo
        enddo
  enddo
!
! ppec = 1 singles out a unique j -> i transition
!
  if ((ppec.eq.1).and.(worder.eq.0)) then
     write(19,*) `ppec = ', ppec
     write(19,*) `Looking at PEC for transition', &
              &   uppec, `->', lopec
        do kk=mntmp,mxtmp
           write(19,7013) temp(kk)
              do Ne=1,Nodens
                 write(19,7015) ED(Ne), pec(uppec,lopec,kk,Ne)
              enddo
        enddo
           write(19,*) `Now loop with a constant density'
        do Ne=1,Nodens
           write(19,7011) ED(Ne)
              do kk=mntmp,mxtmp
                 write(19,7015) temp(kk), pec(uppec,lopec,kk,Ne)
              enddo
        enddo
  endif
!
!        ########################
!        #  Wavelength sorting  #
!        ########################
!
  if (worder.eq.1) then
     print *, `Wavelength ordering requested, &
           &   with worder = ', worder
     write (19,*) `The following PECs are in wavelength order'
     write (19,*) `Tolerance is = ', petolr
     write (19,*) `------------------'
      allocate(wout(order))
        allocate(ipoint(order))
      allocate(jpoint(order))
        CALL Word(wlength,levels,wout,ipoint,jpoint)
        !
        do kk=mntmp,mxtmp
           write(19,*) `Temperature: ', kk
              do Ne=1,Nodens
                 write(19,*) `Density: ', Ne
                    do k=1,order
                       ii = ipoint(k)
                       jj = jpoint(k)
                   if (pec(ii,jj,kk,Ne).gt.petolr) then
                   write(19,7018) ii, jj, wout(k), pec(ii,jj,kk,Ne)
                   endif
                    enddo
              enddo
        enddo

   else
     print *, `No wavelength ordering requested, &
           &   with worder = ', worder
  endif
!
  call cpu_time(finish3)
  print `("Rearranging Wavelength Order = ",f6.2," &
       & minutes.")', (finish3-start3)/60.0
!
  call cpu_time(finish)
  print `("Total Time = ",f6.2," minutes.")', &
       & (finish-start)/60.0
     !
  open(950,file=`lratio',status=`unknown',form=`formatted')
  open(951,file=`dens',status=`unknown',form=`formatted')
  open(952,file=`temp',status=`unknown',form=`formatted')
     !
  write(950,*) `Electron Density | Electron Temperature &
             & | Line Ratio'
  write(950,*) `-----------------------------------------&
             &-----------'
!
! Arrays to be used in Matlab
!
        do Ne=1,Nodens
           write(951,*) ED(Ne)
        enddo
        do kk=mntmp,mxtmp
           write(952,*) temp(kk)
        enddo
        do Ne=1,Nodens
           write(950,*) (lratio(Ne,kk),kk=mntmp,mxtmp)
        enddo
!
!        ########################
!        #  Format specifiers   #
!        ########################
!
  7000 format(` --- RAdiative Transfer Output ---',/,/,` &
   &  - Basic Read of Input - ', /, &
   & 8X, `_____________________________________________')
  7001 format(8X, `| Atomic Number = ', 22X, I3, ` |', / ,&
   & 8X, `| Number of Temperatures = ', 13X, I3, ` |', / ,&
   & 8X, `| Number of Densities = ', 15X, I4, ` |', / ,&
   & 8X, `| Starting at 10^', I1, `K and finishing at 10^',&
   & I2, `K',1X, `|', /, 8X,`| with a mesh increment of ',&
   & F16.4, ` |',/,8X, &
   & `_____________________________________________', /)
  7002 format(`RESERVED FOR SPECIFIC LINES NAMELIST LINE.')
  7003 format(`  - Basic Level Information -  ', /, /,&
   & `Total number of levels in ADF04 file: ', I3, /, /,&
   & `_____________________________________________', /,&
   & `| IND.|   CONFIGURATION     |  ENERGY (eV)  |', /,&
   & `_____________________________________________')
  7004 format(`| ',I3,` | '3X,A13,3X,` | ',F12.4,`  |')
  7005 format(`__________________________________________',&
   & /, /, `The new left hand side values from N_i = 1 - ',&
   & I3,` for a particular temperature and density &
   & (normally 1 and 1)',/)
  7006 format(8D12.4)
  7007 format(/, `Looking at the specific line ratio of', I3,&
   & `->',I3,`/', I3,`->',I3, `over temperatures and densities.')
  7008 format(/, `-------------- Density ...', 4X, I3, &
   & `( ',ES11.4,`)')
  7009 format(2X,E10.4,4X,F20.16)
  7010 format(/, `-------------- Temperature ...', 4X, I3,&
   & `( ',ES11.4,`)')
  7011 format(3X,`Density',1X,99ES11.4)
  7012 format(999ES11.4)
  7013 format(3X,`Temper.',1X,99ES11.4)
  7014 format(10X,I3,1X,ES11.4,2X,ES11.4)
  7015 format(1X,2ES11.4)
  7016 format(1X,I3,2X,I3,3X,A13,1X,`->',1X,A13,1X,F7.4,2X, &
   & F7.4,1X,F14.2)
  7017 format(F14.2,2X,I3,3X,ES11.4,2X,ES11.4)
  7018 format(25X,I3,2X,I3,F14.2,2X,ES11.4)
!
end program

Subroutine Word(wave,lev,wstore,indi,indj)
!
! Read in wavelength(lev,lev) and sort.
!
  integer :: i,ii,j,jj,ord,icount,jcount
  real*8, intent(in) :: wave(lev,lev)
  real*8, intent(out) :: wstore((lev*(lev-1))/2)
  integer, intent(out) ::  indi((lev*(lev-1))/2), &
                       &    indj((lev*(lev-1))/2)

  ord = (lev*(lev-1))/2

do i=2,lev
  do j=1,(i-1)
   jcount = 1
     do ii=2,lev
       do jj=1,(ii-1)
          if ((i.eq.ii).and.(j.eq.jj)) then
             continue
          else
            !
            if (wave(i,j).gt.wave(ii,jj)) then
               jcount = jcount + 1
            else
               continue
            endif
            !
          endif
       enddo
     enddo
   !
   wstore(jcount) = wave(i,j)
   indi(jcount) = i
   indj(jcount) = j
   !
  enddo
enddo

End Subroutine Word

\end{verbatim}
\end{appendices}