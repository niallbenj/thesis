% Chapter 1

\chapter{Introduction} % Main chapter title

\label{cha:intro} % For referencing the chapter elsewhere, use \ref{Chapter1} 

\lhead{Chapter 1. \emph{Introduction}} % This is for the header on each page - perhaps a shortened title

%----------------------------------------------------------------------------------------

\section{Motivation}\label{sec:int_motivation}
Our Universe is simply a collection of matter, comprised of particles such as electrons, protons, neutrons and their antiparticles. From these, more complex structures can be constructed, for example atoms and molecules. When particles and their corresponding antiparticles collide, they annihilate to produce gamma ray photons, through processes that obey the laws of conservation of momentum and energy. Over time, the majority of this matter has been converted into energy, and has resulted in the formation of the most abundant elements in the Universe, hydrogen and helium. These elements can undergo fusion to create stars and also produce new elements such as carbon, oxygen, and even heavier elements.

This basic evolution of matter has since produced phenomenal stellar-objects and events, generally classified by their composition of particular atoms and molecules. In this thesis, we focus our attention primarily on various classes of Supernovae (SNe). They are referred to by their year of explosion, and are classified as being either Type {\sc i} or Type {\sc ii} from their respective spectra. A Type {\sc i} SNe is hydrogen deficient, whereas a Type {\sc ii} contains a number of hydrogen lines from the Balmer series. There are further subcategorizations depending on the observation of particular Si {\sc ii} or He {\sc i} lines, but we will not go into such detail here. When there is a binary system of two stars, these macroscopic events can occur from mechanisms such as thermal runaway - when a white dwarf accretes large amounts of material, heating up its core and initiating nuclear fusion reactions - and core collapse - where the nuclear fusion reactions taking place cannot stabilize the mass of the core and gravitational collapse occurs - both are considered to be the most plausible explanations to date. Isotopes as heavy as Ni$^{56}$ from nucleosynthesis are formed from fusion reactions and decays through the path Ni$^{56} \rightarrow ~ $Co$^{56}\rightarrow~ $Fe$^{56}$ as the explosion occurs. These furious and violent processes then produce supernovae remnants where the ejecta travels at large velocities and contribute heavier elements into the interstellar medium. Therefore, the knowledge of these atomic systems is crucial to understanding important properties of SNe.

To determine the chemical composition of these objects, ground and space-based instruments produce spectra from the radiation that has been emitted. Some of the well known ground-based instruments include the Very Large Telescope in Chile, operating between 300 nm - 20 $\mu$m, and a collection of instruments at various wavelengths in Mauna Kea - Hawaii. Space-based telescopes include the Hubble Space Telescope (year: 1990 / bands: UV, NIR, Visible),  XMM-Newton (year: 1999 / bands: 1 - 120 \AA), and Chandra (year: 1999 / bands: 0.1 - 10 keV). More recently, ASTROSAT has been launched in 2015 from India with multiple instruments on board for wavelengths in soft and hard X-Ray, visible and UV regimes. Looking towards future projects, the James Webb Space Telescope is due for launch in 2018 and will provide even higher resolution in the infrared regimes, and will replace current missions such as the Hubble Space Telescope. These types of missions therefore present excellent opportunities for atomic and astrophysicists worldwide.

To understand what is happening with these macroscopic events, an insight into the microscopic interactions that occur is an essential component. It is therefore our goal to provide the necessary atomic data required to perform spectral modelling. Sophisticated computer codes such as {\sc cloudy} \citep{1998PASP..110..761F}, {\sc xstar} \citep{2001ApJS..134..139B}, and {\sc tlusty} \citep{1995ApJ...439..875H} are currently being implemented to simulate various conditions in order to compare with observational data. We also provide an alternative computer program, {\sc collr-v1.0} written in {\sc fortran90}. Currently, the code only considers an isolated atom approximation and tracks the transitions occurring at given electron temperatures and densities, but additional features are to be included in the future. See Chapter \ref{cha:spectral} for a more detailed description of the theory, and Appendix \ref{app:radt} for a working version of the computer program.

The five most abundant Fe-peak elements are iron, nickel, chromium, manganese, and cobalt, and their abundances are calculated in large star samples \cite{2014AJ....148...67J}, or the Large Magellanic Cloud \cite{2012ApJ...746...29C} as a few examples. Fe and Ni are among the two most studied species, and the remaining three are often overlooked. Their importance is clear from the literature, as various Fe lines can be seen in a multitude of astrophysical objects such as $\eta$ Carinae \citep{2000A&A...361..977J}, the UV spectrum of Quasars \citep{2001ApJS..134....1V}, the optical spectrum of planetary nebula \citep{1981ApJ...248..569S, 1993ApJ...410..430K}, Seyfert galaxies \citep{1985ApJ...297..166O,1999ApJS..125..317L}, and Bp stars \citep{2007A&A...466.1083H} to name just a few. Similarly, Ni is usually a good probe to investigate typical properties of SNe \cite{2001ApJ...547..988M, 2009MNRAS.396.1659M}, and emission lines can be seen in the Orion Nebula \cite{1992ApJ...389..305O}, Seyfert Galaxies \cite{1990ApJ...352..561O}, various stars \cite{2001MNRAS.328..291T, 2004A&A...418.1073W}, and gaseous nebulae \cite{1995A&A...294..555L}.

A number of databases are accessible that contain data for a diverse range of processes involving atomic and molecular systems, and some that are more tailored for particular interactions. One example is the worldwide collaboration of the Opacity Project \citep{1992RMxAA..23..107C,1993A&A...275L...5C}. Rosseland-mean opacities are often implemented when studying stellar structure and evolution, which require knowledge of multiple ionization stages for multiple species. Another extremely beneficial and relevant database is the TOPbase Iron project, with the main focus on accurate data for these discussed Fe-peak species. One final example is the OPEN-ADAS database which contains numerous works for both fusion and astrophysics research. A major driving force is that this data is often used within the corresponding ADAS computer code by the members of the project.

In order to facilitate these modelling codes and also the wider astrophysics and plasma physics communities, we focus our attention specifically on single species during each calculation. The ions of interest in this thesis are S$^{9+}$ and Ar$^{2+}$, which are observed in multiple astrophysical objects, but the major work has been carried out for Co$^{2+}$ and Co$^{+}$. These latter ions have received a lot of attention, and yet, have been poorly studied until recently. We will provide the motivation to perform these calculations for all of these systems in the introductions of their respective Chapters \ref{cha:sulphur}, \ref{cha:argon}, and \ref{cha:cobalt}. 

We apply well known and documented theoretical approaches such as the $R$-matrix method and the configuration-interaction method to consider the interaction of atomic systems with electrons and photons. Quantities such as energy levels, transition probabilities, and oscillator strengths are often analyzed first for an idea of how accurate the description of the system is. There are a number of computer packages based around this theory, but in this thesis we will try to convey that any of these methods will provide meaningful results if used in the correct manner. 

Advances in technology have recently permitted us to perform larger and larger theoretical calculations, so that it is possible to look at these Fe-peak species. The problem with this particular group is that they are known as open d-shell systems. A consequence of this is that they give rise to hundreds and even thousands of target states to be included into the wavefunction representation, which is extremely challenging in terms of computation. The spectra that we obtain, and that are required, are often represented on an energy grid which must be properly resolved, and this often requires tens, or hundreds of thousands of energy points. We avail of our local computing network at Queen's, and also supercomputing time to distribute the workload across multiple processors 

The remainder of this work will be detailed as follows. Chapter \ref{cha:many} lays the foundation by introducing the time independent Schr\"odinger equation and appropriate forms for the total energy operator. Methods for optimized wavefunctions are outlined and implemented into computer packages such as {\sc civ3} and {\sc grasp0}. Useful measurables that can be calculated with accurate wavefunctions are defined, such as $A$-values and oscillator strengths.

Chapter \ref{cha:rmatrix} benefits from the target description in Chapter \ref{cha:many} and extends the basis set to include an extra electron. Similarly, transitions are calculated again between an initial and final state wavefunction for a variety of processes, and we detail the $R$-matrix method for describing atomic interactions with electrons and photons. Configuration space is separated into an internal and external region such that the scattered electron can be treated independently from the remaining electrons in the target wavefunction. A general overview of the codes is also provided in this Chapter to describe the main routines carried out.

Chapter \ref{cha:spectral} contains the basics required for the spectral analysis of the data produced. A non-local thermodynamic equilibrium approach is considered to determine the populations of an isolated atom, and comparisons are made between previous work and our current computer package {\sc collr-v1.0}. An extension to manipulate the photoionization cross-sections for the calculation of recombination rates are useful for radiative transfer simulations, and has also been provided.

Chapter \ref{cha:sulphur} details the first system of interest, S$^{9+}$. We compute photoionization cross-sections from the lowest five initial states of S$^{8+}$ wavefunctions into all allowed states. Detailed spectra are presented across the low energy region between photon energies of 30 - 50 eV. Also produced are the level resolved contributions compared with the available data in OPEN-ADAS. Resonances are identified that converge onto the lowest eight levels and are presented in a tabular form for some of the dominant contributions.

In Chapter \ref{cha:argon} we look at another important system Ar$^{2+}$. The photoionization process is carried out this time for the lowest three levels of Ar$^{+}$, and the initial state wavefunction is statistically weighted according to the split $^2$P$^{\rm o}$ levels. We investigate the results obtained from two different basis sets and $R$-matrix suites of codes, and also compare with other theoretical and experimental work. An extension to look at the L$_2$-shell photoionization of a 2p electron has also been carried out and compared with a recent experimental calculation.

Chapter \ref{cha:cobalt} contains the major results in this thesis for two processes. The first is the photoionization of Co$^+$ and level resolved contributions are calculated to obtain corresponding radiative rates. The second process is the electron-impact excitation calculation, and comparisons are carried out with an extremely recent piece of work. These data sets are then incorporated into the computer code {\sc collr-v1.0} to produce useful temperature and density diagnostics for typical conditions of SNe.

Chapter \ref{cha:conclusions} contains the conclusions and future work proposed. All the results are summarized from Chapters \ref{cha:sulphur}, \ref{cha:argon}, and \ref{cha:cobalt}, noting the major results. Due to the importance of these Fe-peak species, we have also proposed to look at the photoionization of neutral Ni, as well as other neutral and lowly ionized elements. This is a follow up to the work conducted, and also suggested by \citet{clarathesis}.


%----------------------------------------------------------------------------------------


